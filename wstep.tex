\section{Wstęp}
\label{sec:wstep}

\subsection{Cel projektu}
\label{sub:celprojektu}

Celem projektu jest implementacja rozwiązania pozwalającego na rozpoznawanie obiektów na obrazach w reprezentacji zdarzeniowej. Idea reprezentacji zdarzeniowej polega na zaznaczaniu pikseli, dla których nastąpiła pewna zmiana w obrazie wejściowym kamery. Współrzędne wraz z typem zmiany (wzrost lub spadek wartości piksela) tego piksela następnie są wysyłane do zewnętrznego urządzenia, które przetwarza te dane. Metoda ta pozwala na zmniejszenie ilości danych, które potrzebne są do przesłanie z kamery i przetworzenia w innym urządzeniu. Projekt zakłada, że zapisane dane z reprezentacji zdarzeniowej są następnie przetworzone przez sieć neuronową zaimplementowaną w~programie \textit{MATLAB}.


\subsection{Proponowane rozwiązanie}
\label{sub:proponowanerozwiazanie}

Zaproponowane i sprawdzone rozwiązanie wymaga wstępnego przetworzenia zarejestrowanych danych. Polega ono na agregowaniu współrzędnych pikseli z \(10 \si{us}\). Dane te następnie formatowane jako obraz o rozmiarze \(40\times40\si{px}\) oraz jako wektor, którego kolejne elementy są kolejnymi elementami stworzonego obrazu. Stworzone w ten sposób wektory są danymi wejściowymi sieci neuronowej. Dane wyjściowe są zapisywane w~postaci wektora one-hot. Jest to wektor, który ma na jednym miejscu jedynkę, a na pozostałych zera. Tak przygotowane dane poddane są uczeniu z nauczycielem przez głęboką sieć neuronową, ponieważ rozważany problem jest problemem klasyfikacji. Wykorzystane narzędzia to funkcje do uczenia sieci neuronowej z wykorzystaniem autoenkodera zawarte w narzędziu programu \textit{MATLAB - Neural Network Toolbox}. Użyte zostały dwa autoenkodery, które następnie wspólnie z warstwą wyjściową typu \textit{softmax} tworzą całą głęboką sieć neuronową. Zbiór wejściowy podzielono na trzy części: dane do uczenia, walidacji i testowania. W kontekście alternatywnego rozwiązania okazało się prostsze w implementacji. Uzyskano zadowalające efekty, zatem zaproponowane rozwiązanie jest skuteczne i może być z powodzeniem wykorzystywane w procesie rozpoznawania cyfr.


\subsection{Alternatywne rozwiązania}
\label{sub:alternatywnerozwiazania}

Alternatywne rozwiązanie zakładało implementację z użyciem impulsowych sieci neuronowych. Sieci te stanowią stosunkowo nowe podejście, a sposób działania bardziej zbliżony do rzeczywistych procesów zachodzących w mózgu. Lepiej też naśladują naturalne neurony. Bazują na sposobie przekazywania przez nie informacji, jaką jest impuls elektryczny. Jak zostało napisane w \cite{impuls2}, sieci te lepiej nadają się do zadań dynamicznych.
Podejście to wydaje się być idealnym rozwiązaniem do postawionego tu problemu. Naturalne neurony generują impulsy, które są przekazywane dalej \cite{impuls}. Tutaj sieć reagowałaby na impulsy, czyli zmiany położenia obrazu wejściowego, tzw. zdarzenia. W~dostępnej bazie każda kolejna dana generuje nowe zdarzenia, a więc impulsy, wynikające ze zmiany położenia badanego obiektu. 
Pomimo faktu, iż sieć ta idealnie nadałaby się do postawionego problemu, nie zdecydowano się na jej implementację. To stosunkowo nowe podejście, dlatego nie znaleziono gotowych bibliotek, które mogłyby ułatwić proces uczenia, ponieważ napisanie funkcji do uczenia tej sieci to temat obszerny i wykraczający poza możliwości zadania projektowego. Dodatkowo do symulacji tego typu sieci potrzebna jest duża moc obliczeniowa \cite{impuls2}, co wiąże się z koniecznością dostępu do komputera spełniającego pewne wymagania sprzętowe.