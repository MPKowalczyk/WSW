\documentclass[11pt,a4paper,titlepage]{article}
\usepackage[utf8]{inputenc}
\usepackage{amssymb}
\usepackage[english,polish]{babel}
\usepackage[T1]{fontenc}
\usepackage{polski}
%\usepackage[math,light]{anttor}
\usepackage[light]{anttor}
\usepackage{amsmath}
\usepackage{amsfonts}
\usepackage{graphicx}
\usepackage{siunitx}
%\usepackage{subcaption}
\usepackage{sidecap}
%\usepackage{wrapfig}
\usepackage{epstopdf}
\usepackage{booktabs}
\usepackage{forloop}
\usepackage[left=3cm,right=3cm,top=3cm,bottom=3cm]{geometry}
\usepackage[framed,numbered,autolinebreaks]{mcode}
\usepackage[colorlinks=false,urlcolor=blue,citecolor=green]{hyperref}
\usepackage{fancyhdr}
\usepackage{lastpage}
\usepackage{array}
\usepackage{hhline}
\usepackage{svg}
\usepackage{multirow}
\usepackage{enumerate}%[I], numerki, [(a)]
\usepackage{float}
%\usepackage{courier}
%ustawienie poziomów wypunktowania do wyboru: $\bullet$, $\cdot$, $\diamond$, $-$, $\ast$ and $\circ$ 
\renewcommand{\labelitemi}{$\diamond$}
\renewcommand{\labelitemii}{$\bullet$}
\renewcommand{\labelitemiii}{$-$}
\renewcommand{\labelitemiv}{$\ast$}

%Figure numbering
\usepackage{chngcntr}
\counterwithin{figure}{section}
\counterwithin{equation}{section}

\newcommand*{\captionsource}[2]{%
  \caption[{#1}]{%
    #1%
    \\\hspace{\linewidth}%
    \textbf{Źródło:} #2%
  }%
}

\AtBeginDocument{

	\renewcommand{\tablename}{Tabela}

	\renewcommand{\figurename}{Rys.}
}

%tabelki
\usepackage{tabularx}
\newcolumntype{A}{>{\centering\arraybackslash}X}
\newcolumntype{B}{>{\centering\arraybackslash} m{0.4\textwidth} }

% --- < bibliografia > ---
\usepackage[
style=numeric,
sorting=none,
% Zastosuj styl wpisu bibliograficznego właściwy językowi publikacji.
language=auto,
autolang=other,
% Zapisuj datę dostępu do strony WWW w formacie RRRR-MM-DD.
urldate=edtf,
seconds=true,
% Nie dodawaj numerów stron, na których występuje cytowanie.
backref=false,
% Podawaj ISBN.
isbn=true,
% Nie podawaj URL-i, o ile nie jest to konieczne.
url=false,
% Ustawienia związane z polskimi normami dla bibliografii.
maxbibnames=3,
% Jeżeli używamy BibTeXa:
backend=biber
]{biblatex}
% --- < bibliografia > --- Koniec

\usepackage{csquotes}
\DeclareQuoteAlias{croatian}{polish} % Ponieważ `csquotes` nie posiada polskiego stylu, można skorzystać z mocno zbliżonego stylu chorwackiego.

\addbibresource{bibliografia.bib}

\pagestyle{fancy}
\fancyhf{}
\fancyhead[R]{Rozpoznawanie obiektów na obrazach zdarzeniowych}
\fancyfoot[R]{Wieloprocesorowe Systemy Wizyjne}
\fancyhead[L]{M. Kowalczyk i J. Piechota}     
\fancyfoot[L]{Strona \thepage \hspace{1pt} z\hspace{1pt} \pageref*{LastPage}}    
\renewcommand{\headrulewidth}{1pt}
\renewcommand{\footrulewidth}{1pt}

\begin{document}
\begin{titlepage}

\newcommand{\HRule}{\rule{\linewidth}{0.5mm}} % Defines a new command for the horizontal lines, change thickness here

\center % Center everything on the page
 
%----------------------------------------------------------------------------------------
%	HEADING SECTIONS
%----------------------------------------------------------------------------------------

\textsc{\LARGE Akademia Górniczo - Hutnicza im. Stanisława Staszica w Krakowie}\\[0.5cm]
\includegraphics[scale=0.6]{agh}\\[1cm] % Name of your university/college
\textsc{\Large Wydział Elektrotechniki, Automatyki, Informatyki i Inżynierii Biomedycznej}\\[0.5cm] % Major heading such as course name
\textsc{\large Kierunek: Automatyka i robotyka}\\[0.5cm] % Minor heading such as course title

%----------------------------------------------------------------------------------------
%	TITLE SECTION
%----------------------------------------------------------------------------------------

\HRule \\[0.4cm]
{ \huge \bfseries Wieloprocesorowe Systemy Wizyjne\\[1cm]Rozpoznawanie obiektów na obrazach w~reprezentacji zdarzeniowej}\\[0.4cm] % Title of your document
\HRule \\[1cm]%[3.5cm]
 


%----------------------------------------------------------------------------------------
%	DATE SECTION
%----------------------------------------------------------------------------------------

{\large styczeń 2018}\\[1.5cm] % Date, change the \today to a set date if you want to be precise

%----------------------------------------------------------------------------------------
%	LOGO SECTION
%----------------------------------------------------------------------------------------

%\includegraphics[height=70mm]{agh.jpg}%\\[1cm] % Include a department/university logo - this will require the graphicx package
%----------------------------------------------------------------------------------------
%	AUTHOR SECTION
%----------------------------------------------------------------------------------------

\begin{flushleft}
\Large
\emph{Wykonali:}\\
Marcin Kowalczyk\\
Jadwiga Piechota\\[1cm]

% If you don't want a supervisor, uncomment the two lines below and remove the section above
 \emph{Prowadzący:}\\
dr inż. Mirosław Jabłoński\\[3cm] % Your name
 
\end{flushleft}
%----------------------------------------------------------------------------------------
\end{titlepage}
\clearpage
\setcounter{page}{2}

\begin{center}
\begin{figure}[H]
	\centering
	\includegraphics[scale=1]{obrazki/wstep}
\end{figure}
\end{center}

\section*{Streszczenie}
Celem projektu było zaprojektowanie i implementacja sieci neuronowej do rozpoznawania obiektów na obrazach w reprezentacji zdarzeniowej. Skorzystano z bazy danych \textit{MNIST-DVS}, która zawiera obrazy cyfr zapisane w tej reprezentacji. W celu zmniejszenia ilości danych potrzebnych do przetworzenia oraz poprawy jakości obrazów wykonano proste przetwarzanie wstępne. Dane zostały następnie przekonwertowane do postaci, która może posłużyć do uczenia sieci neuronowej. Na koniec sprawdzono skuteczność w zależności od wielkości sieci.
 
\clearpage
\tableofcontents
\clearpage

\section{Wstęp}
\label{sec:wstep}

\subsection{Cel projektu}
\label{sub:celprojektu}

Celem projektu jest implementacja rozwiązania pozwalającego na rozpoznawanie obiektów na obrazach w reprezentacji zdarzeniowej. Idea reprezentacji zdarzeniowej polega na zaznaczaniu pikseli, dla których nastąpiła pewna zmiana w obrazie wejściowym kamery. Współrzędne wraz z typem zmiany (wzrost lub spadek wartości piksela) tego piksela następnie są wysyłane do zewnętrznego urządzenia, które przetwarza te dane. Metoda ta pozwala na zmniejszenie ilości danych, które potrzebne są do przesłanie z kamery i przetworzenia w innym urządzeniu. Projekt zakłada, że zapisane dane z reprezentacji zdarzeniowej są następnie przetworzone przez sieć neuronową zaimplementowaną w~programie \textit{MATLAB}.


\subsection{Proponowane rozwiązanie}
\label{sub:proponowanerozwiazanie}

Zaproponowane i sprawdzone rozwiązanie wymaga wstępnego przetworzenia zarejestrowanych danych. Polega ono na agregowaniu współrzędnych pikseli z \(10 \si{us}\). Dane te następnie formatowane jako obraz o rozmiarze \(40\times40\si{px}\) oraz jako wektor, którego kolejne elementy są kolejnymi elementami stworzonego obrazu. Stworzone w ten sposób wektory są danymi wejściowymi sieci neuronowej. Dane wyjściowe są zapisywane w~postaci wektora one-hot. Jest to wektor, który ma na jednym miejscu jedynkę, a na pozostałych zera. Tak przygotowane dane poddane są uczeniu z nauczycielem przez głęboką sieć neuronową, ponieważ rozważany problem jest problemem klasyfikacji. Wykorzystane narzędzia to funkcje do uczenia sieci neuronowej z wykorzystaniem autoenkodera zawarte w narzędziu programu \textit{MATLAB - Neural Network Toolbox}. Użyte zostały dwa autoenkodery, które następnie wspólnie z warstwą wyjściową typu \textit{softmax} tworzą całą głęboką sieć neuronową. Zbiór wejściowy podzielono na trzy części: dane do uczenia, walidacji i testowania. W kontekście alternatywnego rozwiązania okazało się prostsze w implementacji. Uzyskano zadowalające efekty, zatem zaproponowane rozwiązanie jest skuteczne i może być z powodzeniem wykorzystywane w procesie rozpoznawania cyfr.


\subsection{Alternatywne rozwiązania}
\label{sub:alternatywnerozwiazania}

Alternatywne rozwiązanie zakładało implementację z użyciem impulsowych sieci neuronowych. Sieci te stanowią stosunkowo nowe podejście, a sposób działania bardziej zbliżony do rzeczywistych procesów zachodzących w mózgu. Lepiej też naśladują naturalne neurony. Bazują na sposobie przekazywania przez nie informacji, jaką jest impuls elektryczny. Jak zostało napisane w \cite{impuls2}, sieci te lepiej nadają się do zadań dynamicznych.
Podejście to wydaje się być idealnym rozwiązaniem do postawionego tu problemu. Naturalne neurony generują impulsy, które są przekazywane dalej \cite{impuls}. Tutaj sieć reagowałaby na impulsy, czyli zmiany położenia obrazu wejściowego, tzw. zdarzenia. W~dostępnej bazie każda kolejna dana generuje nowe zdarzenia, a więc impulsy, wynikające ze zmiany położenia badanego obiektu. 
Pomimo faktu, iż sieć ta idealnie nadałaby się do postawionego problemu, nie zdecydowano się na jej implementację. To stosunkowo nowe podejście, dlatego nie znaleziono gotowych bibliotek, które mogłyby ułatwić proces uczenia, ponieważ napisanie funkcji do uczenia tej sieci to temat obszerny i wykraczający poza możliwości zadania projektowego. Dodatkowo do symulacji tego typu sieci potrzebna jest duża moc obliczeniowa \cite{impuls2}, co wiąże się z koniecznością dostępu do komputera spełniającego pewne wymagania sprzętowe.
\section{Analiza złożoności i estymacja zapotrzebowania na zasoby}
\label{sub:analiza}

Zadanie projektowe zostało podzielone na dwie części. Pierwsza z nich dotyczyła odpowiedniego przygotowania danych, czyli wybrania odpowiednich zdarzeń z bazy \textit{MNIST-DVS} i dostosowania ich do aktualnych potrzeb. Dane te zostały następnie użyte podczas właściwego działania algorytmu, czyli uczenia i testowania sieci neuronowej.

\subsection{Przygotowanie danych}
\label{sub:przygotowanie}

Cała baza \cite{baza} zawierała bardzo dużo informacji. Nie było konieczności użycia tak dużej liczby wartości, dlatego zdecydowano się na wybranie około 10\%. Liczba ta okazała się wystarczająca do zobrazowania wyników istotnych z punktu widzenia założonych celów.
W celu znalezienia zbioru uczącego, napisano skrypt w programie \textit{MATLAB 2017b}. Ta część zadania nie wymagała specjalnych zasobów obliczeniowych. Podstawowe wymagania systemowo - sprzętowe, potrzebne do poprawnego działania programu \textit{MATLAB 2017b}, znalezione w \cite{wymagania_Matlab}, są następujące:

\begin{itemize}
\item system operacyjny: Windows 7 SP1 / Windows 8.1 / Windows 10
\item procesor: Intel lub AMD x86-64 (rekomendowane wsparcie dla instrukcji AVX2)
\item dysk: 4 - 6 Gb dla instalacji, 2 Gb wolnego miejsca dla prawidłowego działania programu
\item RAM: co najmniej 2 Gb
\item grafika: nie jest wymagana żadna konkretna karta graficzna, jednak polecana to taka, która ma wsparcie dla OpenGL 3.3 z 1 Gb pamięci GPU.
\end{itemize}

Zasoby pamięciowe, jakie dodatkowo musi posiadać komputer PC, na którym działa program, muszą pomieścić oryginalną bazę, która zajmuje około 9 Gb miejsca. Jednak już po przetworzeniu jej przez \textit{MATLAB} zajmuje dużo mniej miejsca - to około 200 Mb.
Aplikacja musi poradzić sobie z dużą ilością danych - na bazę składa się około 100 000 plików,  przedstawiających zdarzenia przypadające na pewne przesunięcie danej liczby. Pliki te są zapisane w formacie .aedat, stworzonym przez twórców \cite{MNIST_DVS}. Autorzy stworzyli specjalny skrypt, który pozwala na zamianę formatu .aedat na format, który może być odczytany i dalej przetwarzany przez \textit{MATLAB} (format .mat). Wymaga to wykonaniu dodatkowego szeregu instrukcji przy każdorazowym przetwarzaniu pliku z bazy, a wyniki tu generowane są przedstawione w formacie double. To wszystko powoduje dość długi czas trwania obliczeń, dlatego im lepsze parametry ma komputer PC, na którym wykonywane są obliczenia, tym wyniki uzyskiwane są sprawniej. Po wykonaniu obliczeń, uzyskuje się dwie macierze, wykorzystywane później do uczenia sieci. W obu wartościami są liczby boolean. Ponieważ nie ma tutaj żadnych zależności z pozostałymi modułami (baza została przygotowana wcześniej i nie koliduje z procesem uczącym), nie ma znaczenia na jakim komputerze będą wykonywane obliczenia. Możliwości sprzętowe powodują jednak, że efektywność pracy rośnie.

\subsection{Uczenie sieci neuronowej}
\label{sub:uczenie}

Właściwa część algorytmu, czyli uczenie i testowanie sieci neuronowej została również wykonana w programie \textit{MATLAB 2017b}. Wymagania sprzętowo - programowe są tu więc identyczne co te, wymienione w sekcji \ref{sub:przygotowanie}. Jednak tutaj zastosowano dwa różne podejścia do tematu.

Pierwsze podejście zakładało użycie gotowego GUI programu \textit{MATLAB - Neural Network Toolbox}, stworzonego do uczenia i testowania sieci neuronowej. Jedynym wymaganiem był tu dostęp do tego narzędzia z poziomu programu \textit{MATLAB}. 

Drugie podejście polegało na stworzeniu sieci neuronowej za pomocą funkcji programu \textit{MATLAB}. Dodatkowo, w celu przyspieszenia obliczeń, wykorzystano tutaj GPU (z ang. \textit{ graphics processing unit}). Aby taki algorytm mógł działać, niezbędna jest tutaj karta graficzna NVIDIA wyposażona w jednostkę obliczeniową, obsługująca architekturę obliczeniową CUDA (z ang. \textit{Compute Unified Device Architecture}). Takie podejście jest uzasadnione, ponieważ uczenie sieci neuronowych to proces równoległy. Zatem zwiększenie zrównoleglenia obliczeń poprzez użycie dodatkowego procesora znacznie usprawni przebieg działania i spowoduje, że testowanie, polegające na wielokrotnym uczeniu sieci z różnymi parametrami będzie szybsze, a cały proces bardziej efektywny. Wydajność w tym przypadku wzrasta w znacznym stopniu.

W obu podejściach jako wektory wejściowe podane są macierze:

\begin{itemize}
\item danych -  94 490 $\times$ 1600
\item wyjść -  94 490 $\times$ 10 
\end{itemize}
W obu przypadkach w pierwszym podejściu były to wartości logiczne, w drugim zostały zapisane jako typ \textit{double}.
\section{Koncepcja proponowanego rozwiązania}
\label{sub:koncepcja}

Pierwsza część pracy polegała na stworzeniu bazy danych, na których następnie można było uczyć sieć neuronową. Jak zostało wspomniane w sekcji \ref{sub:analiza}, około 10\% danych dostarczonych przez autorów bazy \textit{MNIST-DVS} w źródle \cite{MNIST_DVS} zostało wykorzystanych w~niniejszej pracy. Ponieważ rozważano dwie koncepcje, dane zostały przygotowane do realizacji obu w podobny sposób, jednak na końcu zostały zapisane w innej formie, co zostanie omówione w późniejszej części rozdziału.

Dane zostały wybrane w sposób losowy, odrzucono tylko początkowe wartości, które mogłyby być nieco zaszumione i wprowadzać niepotrzebne rozbieżności. Mając już tak przygotowany zestaw danych, zauważono pewną powtarzalność w przedstawianiu zdarzeń. Otóż zdarzenia przedstawiające zmianę położenia liczby znajdują się w podobnym miejscu (jak już zostało wspomniane, nie cały zestaw danych, czyli zbiór zdarzeń został wzięty pod uwagę). Dzięki temu można było znacząco zmniejszyć wymiary danych wejściowych do rozmiaru 40 $\times$ 40. Taki zabieg przyspieszył i tak już zaawansowane obliczenia.
W dalszej kolejności stworzono tablicę wartości boolean o rozmiarze 40 $\times$ 40 i~przepisano tam wartość 1 w miejscach, gdzie wystąpiły zdarzenia. W ten sposób powstało 94 490 takich tablic (po około 10 000 na jedną cyfrę). Braki wynikają z błędów twórców bazy \textit{MNIST-DVS}, ale zostały wyeliminowane w procesie działania zaprezentowanego algorytmu. Ponieważ uczenie sieci neuronowej zastosowanej w omawianej metodzie następowało z nauczycielem, potrzebne było stworzenie macierzy wyjść. Macierz do tego utworzona miała tyle wierszy, ile tablic wejściowych, zaś kolumn tyle co cyfr, czyli 10. Wartość 1 w danej kolumnie informowała tu z jaką cyfrą mamy w tym przypadku do czynienia - numer kolumny wskazywał cyfrę pomniejszoną o jeden (pierwsza kolumna dla 0).  

Uczenie sieci odbyło się równolegle na dwa sposoby. Pierwszy wymagał przedstawienia danych w postaci wierszowej. Stworzono więc macierz mającą tyle wierszy, ile danych wejściowych i tyle kolumn, ile wartości przypadało na każdą daną, czyli 1600. Drugi sposób pozwalał na stworzenie tablicy celek. Dane wejściowe były przedstawione w taki sposób, że każda celka była tablicą 40 $\times$ 40, czyli jedną daną. Etykiety stworzono bazując na macierzy wyjść. Każda celka przestawiała tu wektor o rozmiarze 10, w którym na odpowiednim miejscu, oznaczającym daną cyfrę, znajdowała się wartość 1. Synchronizację otrzymano poprzez numer celki w danych wejściowych odpowiadał numerowi celki w tablicy wyjściowej zawierającej etykiety.

Pierwsze podejście uczenia sieci zakładało użycie GUI programu \textit{MATLAB - Neural Network Toolbox}, a dokładnie jego części odpowiedzialnej za rozpoznawanie wzorców i~klasyfikację danych - \textit{nprtool}. Po wprowadzeniu tu wartości wejściowych i etykiet, ustala się liczbę neuronów ukrytych i można już przejść do nauki sieci neuronowej. \textit{MATLAB} sam ustala tutaj liczbę danych potrzebnych do uczenia, walidacji i testowania. Program używa tutaj sieci neuronowej z użyciem algorytmu wstecznej propagacji błędów (z ang. \textit{backpropagation}). Algorytm uczenia kończy działanie po udanej sześciokrotnej walidacji. Wyniki testowania można zaobserwować na stworzonych przez GUI statystykach.

Drugie podejście zostało stworzone po to, aby móc zastosować głęboką sieć neuronową. Uczenie tej sieci odbywa się używając macierzy wejściowej i etykiet wyjściowych, czyli jest to uczenie z nauczycielem. Dodatkowo wprowadzone parametry pozwalają na zastosowanie równoległości obliczeń na wielu rdzeniach oraz użycie procesora graficznego GPU. Zastosowanie jest uzasadnione ze względu na dużą liczbę danych wejściowych, co zostało omówione w sekcji \ref{sub:analiza}. To wszystko pozwala na przyspieszenie tego etapu uczenia. Uzasadnione jest tu użycie autoenkodera w celu dodania warstw ukrytych. Użyto tutaj GPU dodając odpowiedni parametr w funkcji \textit{trainAutoencoder}, ustalona została też maksymalna liczba epok. W celu stworzenia pełnej sieci, dodano warstwę wyjściową. Uwzględniając fakt, że jest to problem klasyfikacyjny, stworzono funkcję aktywacji typu \textit{softmax}. Tak nauczoną sieć poddano testowaniu. Wyniki uzyskane nieco różnią się od wyników uzyskanych w podejściu wcześniejszym, co zostanie omówione w sekcji \ref{sub:testowanie}.

% schemat blokowy
\section{Symulacja i testowanie}
\label{sub:testowanie}



\subsection{Modelowanie i symulacja}
\label{sub:modelowanie}


\subsection{Testowanie a weryfikacja}
\label{sub:weryfikacja}


\section{Rezultaty i wnioski}
\label{sub:rezultaty}

Rezultaty otrzymane podczas przetwarzania bazy \textit{MNIST-DVS} pozwalały na sprawne i~efektywne uczenie. Jak już zostało szczegółowo omówione w sekcji \ref{sub:koncepcja}, dane wejściowe zostały przedstawione jako macierze o rozmiarze $40 \times 40$, w których wartość 1 oznacza, że nastąpiło zdarzenie w danym miejscu. Przykładowe dane zostały pokazane na rysunku \ref{fig:zero} i \ref{fig:osiem}.

\begin{figure}[H]
	\centering
	\includegraphics[scale=2]{obrazki/zero}
	\caption{\label{fig:subcaption_example}Przykład niezaszumionych danych wejściowych przedstwiających zdarzenia reprezentujące cyfrę zero.}{\label{fig:zero}}
\end{figure}

\begin{figure}[H]
	\centering
	\includegraphics[scale=2]{obrazki/osiem}
	\caption{\label{fig:subcaption_example}Przykład zaszumionych danych wejściowych przedstwiających zdarzenia reprezentujące cyfrę osiem.}{\label{fig:osiem}}
\end{figure}

\noindent Zdarzenia reprezentujące cyfrę zero idealnie odwzorowują kontury obszaru zainteresowania. Jednak w przypadku cyfry osiem występują pewne szumy. Stanowią one duży problem dla człowieka, aby rozpoznać spośród licznych zdarzeń zdarzenia reprezentujące ruch konturów przedstawionej cyfry. Zaproponowana metoda jednak radzi sobie dobrze z tym problemem i prawidłowo rozpoznaje liczbę. Widać tu przewagę reprezentacji zdarzeniowej nad klasycznymi metodami, gdzie zaszumienie często powodowało, że sieć nie była poprawnie rozpoznać obiektów. Inne przykłady danych wejściowych zostały przedstawione na rysunku \ref{fig:cyfry}.

\begin{figure}[H]
	\centering
	\includegraphics[scale=0.7]{obrazki/zero2}
	\includegraphics[scale=0.7]{obrazki/jeden}
	\includegraphics[scale=0.7]{obrazki/dwa}
	\includegraphics[scale=0.7]{obrazki/trzy}
	\includegraphics[scale=0.7]{obrazki/cztery}
	\includegraphics[scale=0.7]{obrazki/piec}
	\includegraphics[scale=0.7]{obrazki/szesc}
	\includegraphics[scale=0.7]{obrazki/siedem}
\end{figure}
	\newpage
\begin{figure}[H]
	\centering
	
	\includegraphics[scale=0.7]{obrazki/osiem2}
	\includegraphics[scale=0.7]{obrazki/dziewiec}
	\caption{\label{fig:subcaption_example}Przykłady danych wejściowych - po jednym reprezentującym daną cyfrę.}{\label{fig:cyfry}}
\end{figure}

\subsection{GUI Matlaba - funkcja nprtool}
\label{sub:funkcja}

Gotowe dane poddano symulacji używając narzędzia programu \textit{MATLAB - nprtool}. Na zbiorze wejściowym testowano dano z użyciem różnej liczby neuronów ukrytych, w celu znalezienia wartości optymalnej. Wartość ta powoduje odpowiednio maksymalne wytrenowanie sieci, nie powodując przy tym jej przetrenowania. Dane przedstawiające jaki wpływ ma ilość neuronów ukrytych na skuteczność uczenia przedstawia tabela 1.

\begin{center}
Tabela 1: Zależność liczby neuronów ukrytych od skuteczności uczenia sieci.
\begin{figure}[H]
	\centering
	\includegraphics[scale=0.9]{obrazki/wyniki}
\end{figure}
\end{center}

\noindent Widać tutaj dużą zależność pomiędzy ilościami neuronów ukrytych, a skutecznością rozpoznawania cyfr. Przy ilości neuronów 450 zdolność rozpoznawania dla zbioru uczącego osiągnęła prawie 100\%. Dla zbioru testowego, co jest właściwym wynikiem skuteczności działania algorytmu, wskaźnik jakości otrzymał wartość 90.9\%. To wynik uśredniony dla wszystkich dziesięciu cyfr. Odchylenie standardowe danych wynosiło 2.7086\%. Z uwagi na dużą liczbę danych, uzyskano dobrą skuteczność, pomimo iż część danych ze zbioru była mocno zaszumiona i algorytm nie jest w stanie ich rozpoznać. 


\noindent Im więcej neuronów ukrytych, tym uczenie sieci trwa dłużej, nawet około 40 minut. Obciążenie procesora dochodzi tu do 80\%. Jest to więc proces potrzebujący duże zasoby obliczeniowe. Prawdopodobnie dla większej liczby neuronów ukrytych uzyskano by większą skuteczność, jednak rozpoznawanie na poziomie 91\% w reprezentacji zdarzeniowej było wystarczająco dobrym wynikiem.

\subsection{Skrypt do uczenia głębokiej sieci neuronowej}
\label{sub:skrypt}

Sieć neuronowa na wejściu otrzymuje celki o rozmiarze 40 $\times$ 40, czyli łącznie 1600 elementów. Wynika z tego fakt, że liczba neuronów wejściowych jest również równa 1600. Podczas implementacji skryptu wybrano najbardziej optymalną ilość neuronów ukrytych. Zgodnie z wynikami uzyskanym i w sekcji \ref{sub:funkcja} była to liczba 450. Zastosowano tu jednak trochę inne podejście - użyto dwa auto-enkodery - pierwszy posiadający 300, a~drugi 150 neuronów. Na końcu zastosowano warstwę wyjściową typu \textit{softmax}, która jako sygnały wejściowe używa wartości wyjść z drugiego auto-enkoder i pozwala na uzyskanie odpowiedniej liczby wyjść. Schemat całościowy sieci wraz z ilością neuronów w każdej warstwie przedstawiono na rysunku \ref{fig:siec_n}.

\begin{center}
\begin{figure}[H]
	\centering
	\includegraphics[scale=2.1]{obrazki/siec_all}
	\caption{\label{fig:subcaption_example}Schemat zastosowanej głębokiej sieci neuronowej.}{\label{fig:siec_n}}
\end{figure}
\end{center}

\noindent Dokładny opis funkcji użytych do stworzenia i nauczenia sieci znajduje się w sekcji \ref{sub:dokumentacja}. 


\noindent Po wytrenowaniu sieci uzyskane wyniki przedstawiono na wykresie. Interesujące są te, które zostały otrzymane dla zbioru testowego, ponieważ to dla tych danych ustala się wskaźnik jakości algorytmu. Dokładne wyniki przedstawiające rozkład, które cyfry były najczęściej rozpoznawane lub mylone z innymi znajduje się na rysunku \ref{fig:wynik2}.

\begin{center}
\begin{figure}[H]
	\centering
	\includegraphics[scale=0.6]{obrazki/wynik_deep}
	\caption{\label{fig:subcaption_example}Wynik uczenia głębokiej sieci neuronowej, posiadającej 450 neuronów ukrytych. Wyniki pokazane dla zbioru testowego stanowiącego 15\% danych.}{\label{fig:wynik2}}
\end{figure}
\end{center}


\noindent Wykres przedstawia skuteczność rozpoznawania przez sieć cyfr 0 - 9. W celu ułatwienia obliczeń cyfry zostały przeskalowane przez dodanie jedynki (1 - 10). Oś \textit{Target Class} przedstawia rzeczywiste cyfry, zaś oś \textit{Output Class} podaje informację, w jaki sposób algorytm rozpoznał te cyfry (do jakiej grupy je sklasyfikował). W każdym polu, u góry, zaznaczone pogrubioną czcionką znajduje się ilość danych sklasyfikowanych jak para (rzeczywista dana, wyjściowa dana). W tym samym polu na dole obliczony jest stosunek procentowy tych danych w całym zbiorze wejściowym określającym daną cyfrę. Pola oznaczone kolorem szarym przedstawiają zsumowane wyniki pól dla danej kolumny lub rzędu. Ostateczny wynik, uwzględniający zbiór cyfr całościowo, bez względu na rozróżnianie cyfr, znajduje się w prawym, dolnym rogu, oznaczony kolorem niebieskim.


\noindent Średnia wartość skuteczności, a więc przyjęty wskaźnik jakości otrzymał tu wartość 73.1\%. To wynik uśredniony dla wszystkich dziesięciu cyfr. Odchylenie standardowe danych wynosiło 8.0810\%. W przypadku tej metody nie uzyskano tak wysokiej skuteczności jak w poprzedniej, a więc do tego rodzaju danych lepiej sprawdziła się sieć typu \textit{backpropagation}. 


\noindent Zdecydowano się więc na dotrenowanie sieci poprzez użycie metody \textit{backpropagation} dla wszystkich warstw. Uzyskano zadowalające efekty. Wyniki znajdują się na rysunku \ref{fig:wynik3}

\begin{center}
\begin{figure}[H]
	\centering
	\includegraphics[scale=0.6]{obrazki/wynik_deep_tune}
	\caption{\label{fig:subcaption_example}Wynik uczenia głębokiej sieci neuronowej, posiadającej 450 neuronów ukrytych, z użyciem metody \textit{backpropagation}. Wyniki pokazane dla zbioru testowego stanowiącego 15\% danych.}{\label{fig:wynik3}}
\end{figure}
\end{center}


\noindent Średnia wartość skuteczności, a więc przyjęty wskaźnik jakości otrzymał tu wartość wysoką - 93.8\%. To wynik uśredniony dla wszystkich dziesięciu cyfr. Odchylenie standardowe danych wynosiło 1.6753\%. W przypadku tej metody uzyskano najwyższą skuteczność, a więc do tego rodzaju danych lepiej sprawdziła się sieć typu \textit{backpropagation} z dwoma warstwami ukrytymi, mająca taką samą liczbę neuronów ukrytych co w poprzedniej metodzie mającej jedną warstwę. 

Do wykonania tych obliczeń użyte były zarówno zasoby CPU jak i GPU, co pozwoliło zmniejszy czas obliczeń. Obciążenie procesora wynosiło około 70\% maksymalnej wydajności. Uzyskano satysfakcjonujące wyniki, dlatego poprzestano testy na takiej liczbie neuronów.

\subsection{Porównanie metod}
\label{sub:porownanie}

Oba podejścia używają sieci typu \textit{backpropagation} bazującej na metodzie gradientowej obliczania podczas aktualizacji wag. W \textit{MATLAB} do realizacji tego celu służy finkcja \textit{trainscg}. Różnica w obu podejściach polega na różnej liczbie warstw ukrytych - w~pierwszym podejściu była jedna warstwa, w drugim dwie reprezentowane za pomocą stworzenia dwóch auto-enkoderów. Wyniki pokazują, że dwie warstwy sprawdziły się nieco lepiej w przypadku tego zadania. Ilość ukrytych neuronów pozostała taka sama w~obu podejściach, jednak dzielenie procesu uczenia na dwie warstwy okazało się lepszym wyborem. Skuteczność na zbiorze testowym jest blisko 2\% wyższa.
Obie metody pokazują, że istnieje pewna analogia pomiędzy popełnianiem błędów w cyfrach podobnych do siebie pod pewnymi względami. Zauważono, że często algorytm rozpoznawał cyfrę '5'~jako '6' lub '8', a '8' jako '9' lub '5'. Ma to związek z faktem, że cyfry te mają dużo części wspólnych, a reprezentacja zdarzeniowa nie dostarcza pełnej informacji o krawędziach, ale tylko część tej informacji, czyli miejsce, gdzie nastąpiła pewna zmiana. 
\section{Podsumowanie}
\label{sub:podsumowanie}



\section{DODATEK A: Szczegółowy opis zadania}
\label{sub:dodatekA}

\subsection{Specyfikacja projektu}
\label{sub:specyfikacja}

Tematem projektu było rozpoznawanie obiektów na obrazach w reprezentacji zdarzeniowej. Postawione zadanie wymagało znalezienia odpowiednich danych, które zostały przedstawione poprzez następujące po sobie zdarzenia. Podejście to nieco różni się od standardowego, trudno byłby wygenerować zbiór uczący  w ramach realizacji tego zadania, bo to już jest temat na inny projekt. W literaturze poleconej przez prowadzącego znaleziono metodę, której wynikiem była baza \textit{MNIST-DVS}. Baza ta została odnaleziona w \cite{baza} i wykorzystana do uczenia sieci neuronowej. Celem, jaki został postawiony, było nauczenie sieci neuronowej, bazując na zdarzeniowej reprezentacji danych, aby była w~stanie rozpoznać 10 cyfr: 0, 1, 2, 3, 4, 5, 6, 7, 8, 9. Wybór rodzaju sieci oraz platform sprzętowo - programowych prowadzący zostawił autorom projektu. 

\subsection{Szczegółowy opis realizowanych algorytmów przetwarzania danych}
\label{sub:opis}

Sekcja ta będzie rozszerzeniem informacji zawartych już wcześniej w paragrafie \ref{sub:koncepcja}. Zastosowane algorytmy zostaną tu omówione bardziej od strony matematycznej.

\subparagraph{Przygotowanie danych}
\label{sub:przyg}

Przetwarzanie bazy \textit{MNIST-DVS} zostało napisane w programie \textit{MATLAB}. Było to konieczne, ponieważ autorzy artykułu \cite{MNIST_DVS} zastosowali rozszerzenie .aedat. Dostarczyli dodatkowo skrypt \textit{dat2mat}, który pozwalał na zamianę tych danych na pliki z rozszerzeniem .mat, które można już w łatwy sposób podglądnąć i~przetworzyć. W tym przypadku zastosowano przetwarzanie sekwencyjne, tworząc skrypt, który zamieniał rozszerzenie i dostosowywał bazę do realizacji postawionych w projekcie celów.

\subparagraph{Uczenie sieci}
\label{sub:ucz}

Uczenie sieci przebiegało z użyciem wbudowanych funkcji programu \textit{MATLAB}, co znacznie ułatwiło pracę. Dane były przetwarzanie współbieżnie, wykorzystując rdzenie procesora CPU oraz procesor graficzny GPU. Pierwszy sposób, uwzględniający GUI do uczenia sieci neuronowych, używał algorytmu wstecznej propagacji błędów i będzie omówiony w sekcji \ref{sub:wstecz}. Drugi sposób polegał na wykorzystaniu wbudowanych funkcji do uczenia maszynowego i używał głębokich sieci neuronowych do klasyfikacji danych. 

\subparagraph{Algorytm wstecznej propagacji błędów}
\label{sub:wstecz}

To podstawowy algorytm uczenia nienadzorowanego wielowarstwowych sieci neuronowych. Zaletą tej sieci jest to, że wagi można tu wytrenować, znajdując ich optymalny zestaw \cite{back}. Metoda umożliwia modyfikację wag w sieci o architekturze warstwowej, we wszystkich jej warstwach. Po ustaleniu topologii, początkowe wagi są inicjowane tu losowo. Przyjmują bardzo małe wartości. Następnie dla danego wektora uczącego oblicza się odpowiedź sieci, warstwa po warstwie, stosując algorytm spadku gradientowego \cite{back2}. Każdy neuron wyjściowy oblicza swój błąd, który następnie jest propagowany do wcześniejszych warstw. Następnie każdy neuron modyfikuje wagi na podstawie wartości błędu. Dodatkowo jest tu wprowadzona powtarzalność, czyli gdy wszystkie dane wejściowe zostaną już użyte, zmieniana jest ich kolejność i ponownie są wprowadzana do sieci. Proces trwa do momentu zatrzymania się średniego błędu kwadratowego. Jest to proces dość kosztowny obliczeniowo, zwłaszcza kiedy sieć jest rozbudowana.

\subparagraph{Głębokie sieci neuronowe}
\label{sub:glebokie}

Uczenie głębokie sieci neuronowej następuje krok po kroku. Pozwala na stopniowe wyznaczenie wag dla poszczególnych warstw sieci w celu innej reprezentacji cech wspólnych. To poszczególne warstwy reprezentują tu cechy wspólne wzorców uczących i na tej podstawie tworzą reprezentacje bardziej skomplikowanych cech w kolejnych warstwach sieci głębokich. Jest to ulepszona metoda niż wielowarstwowe sieci neuronowe uczone algorytmem propagacji wstecznej, w której w~warstwach oddalonych od wyjścia sieci sieć ma tendencję do dokonywania coraz mniejszych zmian \cite{deep}. Tutaj sieć jest rozbudowywana powoli o kolejne warstwy dopiero wtedy, gdy w~poprzednich warstwach pojawiły się takie cechy. Stosuje się tu sieci typu \textit{auto-encoder}, które używając aproksymacji identycznościowej wykorzystują warstwę ukrytą składającą się z mniejszej ilości neuronów niż ilość wejść lub wyjść z sieci. W przypadku głębokich sieci neuronowych trzeba uważać na to, aby sieć nie dopasowała się zbyt mocno do danych uczących, ponieważ na danych testowych może później niepoprawnie działać.

W przypadku realizacji przestawionego projektu wykorzystano głębokie sieci neuronowe do uczenia nadzorowanego, ponieważ rozważano problem klasyfikacji. Do realizacji głębokiego uczenia użyto tutaj auto-enkoderów, jako funkcja aktywacji posłużyła w~warstwie wyjściowej funkcja typu \textit{softmax}. Dokładny opis procedur znajduje się w~sekcji \ref{sub:dodatekB}.

% literatura

\section{DODATEK B: Dokumentacja techniczna}
\label{sub:dodatekB}

Projekt został napisany w programie \textit{MATLAB 2017b}. Korzystano z gotowego GUI programu \textit{MATLAB - Neural Network Toolbox} oraz innych funkcji z tego \textit{toolbox}. Baza danych potrzebna do realizacji postawionego zadania została pobrana z źródła.
%link do bazy

\subsection{Dokumentacja oprogramowania}
\label{sub:dokumentacja}

Wszystkie funkcje użyte na potrzeby zaimplementowania algorytmu zostały zaczerpnięte z gotowych rozwiązań. Opis i dokumentacja techniczna GUI \textit{Neural Network Toolbox} znajduje się w źródle \cite{gui}, zaś do funkcji użytych w procesie tworzenia głębokich sieci neuronowych w źródle \cite{funkcje}.

Kolejność użytych funkcji przedstawiono na schemacie \ref{fig:flow}.

\begin{figure}[H]
	\centering
	\includegraphics[scale=0.45]{obrazki/flow}
	\caption{\label{fig:subcaption_example}Schemat blokowy użytych funkcji do głębokiego uczenia sieci neuronowej.}{\label{fig:flow}}
\end{figure}

Funkcja \textit{trainAutoencoder} tworzy ukryte warstwy. Jako parametry przyjmuje dane wejściowe, ilość ukrytych warstw oraz parametry określające dynamikę i właściwości wag. Dodatkowo można tu zdecydować, czy obliczenia mają być wykonywane również na procesorze GPU. Funkcja \textit{encode} służy do ekstrakcji cech z ukrytych warstw, a jako parametry przyjmuje dane wejściowe oraz wynik działania autoenkodera. Funkcja \textit{trainSoftmaxLayer} tworzy funkcę aktywacji typu softmax, bazując na cechach zwróconych po wywiłaniu funkcji \textit{encode} oraz etykietach danych wejściowych. Określa się tutaj także maksymalną liczbę epok. Ostatnia funkcja \textit{stack} układa wszystkie warstwy, łącząc stworzone autoenkodery oraz warstwę wyjściową. Funkcja ta tworzy głęboką sieć neuronową, która może być już użyta do testowania.


\subsection{Procedura symulacji, testowania i weryfikacji}
\label{sub:procedura}

Do realizacji projektu nie używano żadnej zewnętrznej platformy sprzętowej. Aby uruchomić aplikację wystarczy komputer PC spełniający wymagania sprzętowo - programowe opisane w sekcji \ref{sub:przygotowanie} z zainstalowanym programem \textit{MATLAB 2017b}. Potrzebny jest także \textit{Neural Network Toolbox}, który będzie wbudowany w przypadku pełnej instalacji programu. W celu przygotowania danych do uczenia można pobrać bazę ze źródła \cite{baza} i uruchomić skrypt \textit{Data$\_$selection} znajdujący się na nośniku CD. Można również wczytać przygotowane już dane w formacie .mat: data$\_$matrix.mat oraz images$\_$double.mat znajdujące się również na nośniku CD. W celu weryfikacji algorytmu należy skorzystać z \textit{Neural Network Toolbox}, wybierając opcję \textit{Pattern recognition and classification}. Jako wejście należy wczytać macierze input$\_$matrix oraz output$\_$matrix. Druga opcja zakłada skorzystanie z głębokiego uczenia z użyciem większej liczby warst ukrytych. W tym celu trzeba uruchomić skrypt \textit{GPU$\_$Mnist} (płyta CD). Należy zaznaczyć, że proces trwa dosyć długo, a liczenie przebiega tu w sposób równoległy i z użyciem procesora graficznego GPU.


\section{DODATEK C: Spis zawartości dołączonego nośnika (płyta CD ROM)}
\label{sub:dodatekC}


Struktura folderów nośnika wygląda następująco:
\begin{itemize}
\item DOC - raport z projektu w formacie PDF
\item SRC - zawiera skrypty źródłowe, Matlab2017b
\begin{itemize}
\item dat2mat - stworzony przez autorów bazy \cite{MNIST_DVS}, pozwala na zmianę rozszerzenie plików .aedat na format .mat
\item Data$\_$selection - skrypt stworzony do przygotowania danych do uczenia sieci neuronowej
\item GPU$\_$Mnist - algorytmy uczenia i testowania głębokiej sieci neuronowej użyciem równoległych obliczeń i procesora GPU
\end{itemize}

\item TEST - znajdują się tu zarówno pliki do uczenia i testowania sieci neuronowej, jak i zapis z \textit{workspace} po wytrenowaniu sieci: z udziałem GUI (wynik$\_$450.mat) oraz auto-enkoderów (gpu$\_$wynik$\_$tune.mat).
\end{itemize}
\section{DODATEK D: Historia zmian}
\label{sub:dodatekD}

\begin{center}
Tabela 2: Historia zmian
\begin{figure}[H]
	\centering
	\includegraphics[scale=0.9]{obrazki/postep}
\end{figure}
\end{center}

\newpage

\begin{center}
\begin{figure}[H]
	\centering
	\includegraphics[scale=1]{obrazki/ocena}
\end{figure}
\end{center}

\appendix
\nocite{*}
\printbibliography
\addcontentsline{toc}{section}{Bibliografia}
\end{document}